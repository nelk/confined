\documentclass {article}
\usepackage{fullpage}

\begin{document}

~\vfill
\begin{center}
  \Large

  A5 Project Proposal

  Title: Escape the Darkness

  Name: Alex Klen

  Student ID: 20372654

  User ID: ayklen
\end{center}
\vfill ~\vfill~
\newpage
\noindent{\Large \bf Final Project:}
\begin{description}
  \item[Purpose]:\\
    To create a game that utilizes and exhibits modern OpenGL shader techniques.

  \item[Statement]:\\
    The game should take place in a scene which exhibits many of the modern techniques expressed in the objectives.
    It should extensively use light and shadow and run at an interactive framerate.

    The idea of the game is to escape a house that the main character finds himself in.

    Several algorithms must be implemented in shaders (and all work together).
    These include shadow mapping, deferred rendering (to support many dynamic light sources), texture and bump mapping, an approximation algorithm for ambient occlusion, and planar mirror reflection.

    This project is interesting and challenging because it involves using several advanced shader techniques at the same time, and the game itself will hopefully look good and be interesting to play.

    I will learn how to effectively use GLSL on modern hardware to achieve quality light and shadow rendering at interactive framerates.

  \item[Technical Outline]:\\
    This project will require implementing several algorithms, all of which have to do with using shaders written in GLSL.
    OpenGL 3+ will be used for its efficient use of buffers.

    One of the most important objectives is to implement deferred shading. Deferred shading is a non-traditional method of shading that speeds up subsequent lighting and shadow passes by letting them perform per-pixel lighting calculations instead of computing it for all geometry \cite{ferkoreal}.
    When forward-rendering the vertex shader will compute lighting parameters (incident, normal, colours) which will be interpolated for each fragment. Deferred shading, however, computes geometric parameters for each pixel and writes them to a texture for later use. Subsequent shading passes can then sample these textures and operate per-fragment only. The benefit of this is clear when you consider shadow mapping with many lights, since when forward rendering you would need to run the shadow mapping algorithm over all geometry for each light and each fragment, whereas with deferred shading you only need to do it for each fragment. Although there are many passes the memory bandwidth of each is quite small.

    The other crucial objective is shadow mapping. This algorithm is well-known and consists of a first pass where geometry is rendered from the perspective of a dynamic light source, and only a depth buffer is written \cite{williams1978casting}. This is done for each light. Then a final pass can perform lighting calculations based off of whether the camera's perspective sees pixels that are farther or closer than the first surface each light sees is. This algorithm is compatible with shadow-mapping and in fact allows you to use many dynamic light sources because they only have to pass over each fragment. There are also extensions to shadow mapping that eliminate visual artifacts present in the basic implementation and allow more realistic penumbrae \cite{dimitrov2007cascaded} \cite{fernando2005percentage}.

    Another interesting objective is to apply an ambient occlusion (approximation) algorithm in shaders. There is more than one fast algorithm for this, including Screen Space Ambient Occlusion \cite{kajalinshaderx7}\cite{mittring2007finding}.

    A few other objectives:
    \begin{itemize}
      \item
        Bump/normal mapping is compatible with deferred shading since perturbed normals can be written in the first pass.

      \item
        Planar mirror reflection can be achieved by rendering to texture the scene once from the perspective of the reflected ray from the camera's perspective off of the mirror surface, and then rendering again from the camera's perspective after applying the rendered texture to the mirror.

      \item
        Basic keyframe animation can be implemented by using multiple mesh files with the same corresponding data (can be exported by modelling applications) and performing interpolation between them.

    \end{itemize}

    \newpage

  %\item[Bibliography]:\\

    %\begingroup
    %\renewcommand{\section}[2]{}%
    \nocite{*}
    \bibliography{biblio}{}
    \bibliographystyle{plain}
    %\endgroup

\end{description}
\newpage


\noindent{\Large\bf Objectives:}

{\hfill{\bf Full UserID: ayklen}\hfill{\bf Student ID: 20372654}\hfill}

\begin{enumerate}
  \item[\_\_\_ 1:]  Unique scene that shows off dynamic shadows with many light sources.

  \item[\_\_\_ 2:]  Interactive first-person controls and interaction with environment (UI).

  \item[\_\_\_ 3:]  Texture Mapping.

  \item[\_\_\_ 4:]  Shadow Mapping implemented correctly.

  \item[\_\_\_ 5:]  Bump Mapping.

  \item[\_\_\_ 6:]  Deferred Shading implemented to allow many dynamic lights.

  \item[\_\_\_ 7:]  Ambient Occlusion using a modern shader technique.

  \item[\_\_\_ 8:]  Sound synchronized with game.

  \item[\_\_\_ 9:]  Planar mirror reflection support using rendering to texture.

  \item[\_\_\_ 10:] Keyframe Animation.
\end{enumerate}

% Delete % at start of next line if this is a ray tracing project
% A4 extra objective:
\end{document}
