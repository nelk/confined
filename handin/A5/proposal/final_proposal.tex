\documentclass {article}
\usepackage{fullpage}

\begin{document}

~\vfill
\begin{center}
  \Large

  A5 Project Final Proposal

  Title: Confined

  Name: Alex Klen

  Student ID: 20372654

  User ID: ayklen
\end{center}
\vfill ~\vfill~
\newpage
\noindent{\Large \bf Final Project:}
\begin{description}
  \item[Purpose]:\\
    To create a game that utilizes and exhibits modern OpenGL shader techniques.

  \item[Statement]:\\
    You find yourself confined to a room and need to escape. The focus of this project is on the graphical techniques, and some amount of gameplay and interactivity will be implemented if time permits.
    The game will take place in a scene which exhibits many of the modern techniques expressed in the objectives.
    It will extensively use lights and (dynamic) shadows and run at an interactive framerate.

    The goal is to implement several algorithms using shaders and OpenGL 3.3+ (that all work together).
    These include shadow mapping, deferred rendering (to support many dynamic light sources), texture and bump mapping, an approximation algorithm for ambient occlusion, and planar mirror reflection.

    This project is interesting and challenging because it involves using several advanced shader techniques at the same time, and the game itself will hopefully look good and be interesting to play.

    I will learn how to effectively use GLSL on modern hardware to achieve quality light and shadow rendering at interactive framerates.

  \item[Technical Outline]:\\
    This project will require implementing several algorithms, all of which have to do with using shaders written in GLSL.
    OpenGL 3.3+ will be used for its efficient buffers, custom rendering pipeline, and multiple render targets (required for deferred shading).

    Although not an objective, implement deferred shading will greatly speed up dynamic lighting. Deferred shading is a method of shading that speeds up subsequent lighting and shadow depth passes by letting them perform per-pixel lighting calculations only instead of computing them for all geometry and all fragments \cite{ferkoreal}.
    When forward-rendering, the vertex shader will compute lighting parameters (incident, normal, colours) which will be interpolated for each fragment. Deferred shading, however, computes geometric parameters for each pixel and writes them to a textures for later use. Subsequent shading passes can then sample these textures and operate per-fragment only. The benefit of this is clear when you consider shadow mapping with many lights, since when forward rendering you would need to run the shadow mapping algorithm over all geometry for each light and each fragment, whereas with deferred shading you only need to do it for each fragment. Although there are many shader passes the memory bandwidth of each is quite small.

    A crucial objective is shadow mapping. This algorithm is well-known and consists of a first pass where geometry is rendered from the perspective of a dynamic light source, and only a depth buffer is written \cite{williams1978casting}. This is done for each light. Then a final pass can perform lighting calculations based off of whether the camera's perspective sees pixels that are farther or closer than the first surface each light sees is. This algorithm is compatible with shadow-mapping and in fact allows you to use many dynamic light sources because they only have to pass over each fragment. There are also extensions to shadow mapping that eliminate visual artifacts present in the basic implementation and allow more realistic penumbrae \cite{dimitrov2007cascaded} \cite{fernando2005percentage}.
    The objective requires dynamic shadow maps to work - a light source shouldn't light geometry that is occluded. Several light source types can be implemented - namely directional lights (no position or falloff and parallel light rays), spot lights (position, direction, angle, and falloff), and point lights (position and falloff - lights in all directions). Shadow maps are the most challenging for point lights as a texture cube map will need to be used.

    Another interesting objective is to apply an ambient occlusion (approximation) algorithm in shaders. There is more than one fast algorithm for this, but this project uses the quick algorithm named Screen Space Ambient Occlusion \cite{kajalinshaderx7}\cite{mittring2007finding}.

    A few other objectives:
    \begin{itemize}
      \item
        Bump/normal mapping, implemented in a way compatible with deferred shading (perturbed normals can be written in the first pass).

      \item
        Motion blur - a post processing effect - meaning the scene is rendered to a texture and then another set of shaders will process this texture to output the final result to screen. Motion blur means when you move your camera the scene will blur objects that appear to be moving quicker. There are a few techniques for this - I will try storing the previous viewing matrix and blurring based off of each pixel position's delta, with correction for frame rate.

      \item
        Planar mirror reflection - rendering to texture the scene once from the perspective of the reflected ray from the camera's perspective, and then rendering again from the camera's perspective after applying the rendered texture to the mirror.

      \item
        Basic keyframe animation - implemented by using multiple mesh files with the same corresponding data (can be exported by modelling applications) and performing interpolation between them.

    \end{itemize}

  \item[Bibliography]:\\

    \begingroup
    \renewcommand{\section}[2]{}%
    \nocite{*}
    \bibliography{biblio}{}
    \bibliographystyle{plain}
    \endgroup

\end{description}
\newpage


\noindent{\Large\bf Objectives:}

{\hfill{\bf Full UserID: ayklen}\hfill{\bf Student ID: 20372654}\hfill}

\begin{enumerate}
  \item[\_\_\_ 1:]  Modelled environment that shows off dynamic shadows with many light sources.

  \item[\_\_\_ 2:]  Interactive first-person controls and interaction with environment (UI).

  \item[\_\_\_ 3:]  Real-time Shadow Maps in shader for dynamic shadows. Standard implementation with comparisons against depth textures generated from the perspective of each light.

  \item[\_\_\_ 4:]  Texture Mapping in shader.

  \item[\_\_\_ 5:]  Bump/Normal Mapping in shader.

  \item[\_\_\_ 6:]  Motion blur implemented as a post-processing shader pass.
    %Deferred Shading pipeline.
    % Switch this: depth-of-field? particle systems? motion blur? perlin noise?

  \item[\_\_\_ 7:]  Ambient Occlusion approximation in shader using Screen Space Ambient Occlusion (SSAO). Decreases ambient light contribution for pixels that are occluded by sampled neighboring pixels (using depths and normals in screen space).

  \item[\_\_\_ 8:]  Sound synchronized with game.

  \item[\_\_\_ 9:]  Planar mirror reflection. Find reflected ray off of mirror surface, render scene to texture from ray's perspective, render scene from camera's perspective while applying texture to mirror geometry.

  \item[\_\_\_ 10:] Keyframe Animation.
\end{enumerate}

% Delete % at start of next line if this is a ray tracing project
% A4 extra objective:
\end{document}
